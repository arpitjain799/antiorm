%
% Slides for given talks:
%
% * 2007-02-26: PyCon 2007, Dallas (Texas), USA
%
%\documentclass[compress,trans]{beamer}
\documentclass{beamer}

\usepackage{graphicx}

\mode<presentation>
{
\usetheme{Singapore} % Singapore, default
  \setbeamertemplate{frametitle}[default][left]
  \setbeamercovered{transparent}
}

\usepackage[english]{babel}
\usepackage[latin1]{inputenc}

\usepackage{times}
\usepackage[T1]{fontenc}

\title{An extension to DBAPI 2.0 \\
for easier SQL queries}
\subtitle{}

\author{Martin Blais}

\institute{}

\date{PyCon 2007}

\subject{dbapiext presentation @ PyCon 2007}

\setcounter{tocdepth}{4}

\newcommand{\todo}[1]{}


\AtBeginSubsection[]
{
  \begin{frame}<beamer>
    \frametitle{Outline}
    \tableofcontents[currentsection,currentsubsection]
  \end{frame}
}


\begin{document}

%-------------------------------------------------------------------------------
\begin{frame}
  \titlepage
\end{frame}

%===============================================================================


%-------------------------------------------------------------------------------
\begin{frame}[fragile]
  \frametitle{Introduction}

  DBAPI's \texttt{Cursor.execute()} method interface is inconvenient to use.

\vfill
  With this work:
  \begin{itemize}
  \item Provide a simple extension that gets rid of the pitfalls
  \item Make it much easier to write queries
  \item A single pure Python module
  \item Support a number of DBAPI implementations
  \item Deals only with query writing
  \end{itemize}

\end{frame}


%% %-------------------------------------------------------------------------------
%% \begin{frame}[fragile]
%%   \frametitle{Motivation (Anti-ORM Activism)}
%% 
%%   \begin{center}
%%     \includegraphics[width=0.8\textwidth]{noorms.png}
%%   \end{center}
%% \end{frame}
%% 

%% %-------------------------------------------------------------------------------
%% \begin{frame}[fragile]
%%   \frametitle{Motivation}
%% 
%%   \begin{quote}
%%     While some like to use ORMs to interface with databases, we prefer to
%%     maintain the full power of the SQL language, and instead focus on making the
%%     building of SQL queries much more convenient.
%%   \end{quote}
%% 
%%   \begin{itemize}
%%   \item The domain-specific mini-language of SQL can never be fully replaced
%%     with an API
%%   \item ORMs hide the important facts: when does data transfer occur?  Does data
%%     get queried multiple times?
%%   \end{itemize}
%% 
%% \end{frame}
%% 

%-------------------------------------------------------------------------------
\begin{frame}[fragile]
  \frametitle{\texttt{Cursor.execute()} interface}

  General form:
\begin{verbatim}
  cursor.execute(<string>, <tuple-or-map>)
\end{verbatim}

\vfill

  For example:

\begin{verbatim}
    INSERT INTO Users (username, email) 
       VALUES ('George', 'g@soros.com');
\end{verbatim}

\begin{verbatim}
  cursor.execute("""
    INSERT INTO %s (%s, %s) VALUES (%%s, %%s)
    """ % ("Users", "username", "email"),
    (var_username, var_email))
\end{verbatim}

\end{frame}


%-------------------------------------------------------------------------------
\begin{frame}[fragile]
  \frametitle{\texttt{Cursor.execute()} interface}

  Be careful with escaping values, here is a common mistake:
\begin{verbatim}
 cursor.execute(
   "INSERT INTO Users (username) VALUES (%s)" % 
   var_username)
\end{verbatim}

  These are \emph{also} incorrect:
\begin{verbatim}
 cursor.execute(
   "INSERT INTO Users (username) VALUES ('%s')" % 
   var_username)

 cursor.execute(
   "INSERT INTO Users (username) VALUES (%s)" % 
   repr(var_username))
\end{verbatim}

\end{frame}


%-------------------------------------------------------------------------------
\begin{frame}[fragile]
  \frametitle{\texttt{Cursor.execute()} interface}

  You \emph{must} let DBAPI do its database-specific escaping of values:
\begin{verbatim}
 cursor.execute(
   "INSERT INTO Users (username) VALUES (%s)",
   (var_username,))
\end{verbatim}

String constants, timestamps, dates, etc. ; Formats vary depending on the
database.

\pause
\vfill
Problems:
\begin{itemize}
\item Two lists of parameters is error-prone
\item You have to provide a tuple or a dict for the argument
\item It does not understand lists 
\item You can't leverage the power of keyword arguments
\end{itemize}

\end{frame}


%-------------------------------------------------------------------------------
\begin{frame}[fragile]
  \frametitle{\texttt{Cursor.execute()} interface}

  When you write real-world queries (instead of Mickey-mouse example
  queries), it gets even messier:
\begin{verbatim}
  cursor.execute("""
    SELECT %s FROM %s WHERE %s > %%s LIMIT %s
    """ % (','.join(columns), "Users", "age", 10)
    (18,))
\end{verbatim}

\begin{itemize}
\item Because of string interpolation, you have to \textbf{double-escape} the
  format specifiers for the escaped values
\item The parameters in the strings are in a \textbf{different order} than the
  function arguments (easy to make mistakes!)
\end{itemize}

\end{frame}




%-------------------------------------------------------------------------------
\begin{frame}[fragile]
  \frametitle{New format specifier (\%S)}

  We provide a new \texttt{execute()} method, which supports a format specifier
  for escaped arguments: \texttt{\%S} (capital S)

\begin{verbatim}
 cursor.execute_f(
   "INSERT INTO Users (username) VALUES (%S)",
   var_username)
\end{verbatim}

  You can now mix vanilla and escaped values in the arguments:
\begin{verbatim}
 cursor.execute_f(
   "INSERT INTO Users (%s) VALUES (%S)",
   "username", var_username)
\end{verbatim}

\end{frame}



%-------------------------------------------------------------------------------
\begin{frame}[fragile]
  \frametitle{Lists are understood}

  Lists are automatically joined with commas:
\begin{verbatim}
 columns = ["username", "email", "age"]
 cursor.execute_f(
   "INSERT INTO Users (%s) VALUES (...)",
   columns, ...)
\end{verbatim}

\vfill
\begin{verbatim}
   INSERT INTO Users 
      ('username', 'email', 'age') 
      VALUES (...)
\end{verbatim}
\vfill

\end{frame}



%-------------------------------------------------------------------------------
\begin{frame}[fragile]
  \frametitle{Lists are understood}

  This also works for escaped arguments:
\begin{verbatim}
 columns = ["username", "email", "age"]
 values = [var_username, var_email, var_age]
 cursor.execute_f(
   "INSERT INTO Users (%s) VALUES (%S)",
   columns, values)
\end{verbatim}

\vfill
\begin{verbatim}
   INSERT INTO Users 
      ('username', 'email', 'age') 
      VALUES ('Warren', 'w@buffet.com', 76)
\end{verbatim}
\vfill

\begin{itemize}
\item Values are escaped individually and then comma-joined
\end{itemize}

\end{frame}


%-------------------------------------------------------------------------------
\begin{frame}[fragile]
  \frametitle{Dictionaries are understood}

  Dictionaries are rendered as required for \texttt{UPDATE} statements:
  \begin{itemize}
  \item Comma-separated \texttt{<name> = <value>} pairs
  \item Values are DB-escaped automatically
  \end{itemize}

\begin{verbatim}
  UPDATE languages 
    SET id = 3, brazil = 'portuguese'
\end{verbatim}
\vfill

\begin{verbatim}
  values = {"id": 3, 
            "brazil": "portuguese"}

  cursor.execute_f("UPDATE languages SET %S", 
                   values)
\end{verbatim}
\vfill


(Suggestion by D. Mertz)

\end{frame}



%-------------------------------------------------------------------------------
\begin{frame}[fragile]
  \frametitle{Positional and Keywords Arguments}

  Positional and keyword arguments can be used simultaneously:

\begin{verbatim}
  cursor.execute_f("""

     SELECT %(table)s FROM %s 
       WHERE id = %(id)S

  """, column_names, table=tablename, id=42)
\end{verbatim}

\begin{itemize}
\item You can recycle arguments this way \\
(i.e. a table or column name that occurs multiple times)
\end{itemize}


\end{frame}



%-------------------------------------------------------------------------------
\begin{frame}[fragile]
  \frametitle{Performance and Remarks}

  \begin{itemize}
  \item The extension only massages your query in a form that can be digested by
    DBAPI's \texttt{Cursor.execute()}

  \item I lied slightly in my examples, you have to use it like this:
\begin{verbatim}
   execute_f(cursor, """
      ...
\end{verbatim}

  \item We cache as much of the preprocessing as possible \\
(similar to \texttt{re}, \texttt{struct})
    \begin{itemize}
    \item You can cache your queries at load time with \texttt{qcompile()}.
    \end{itemize}
  \end{itemize}


\end{frame}



%-------------------------------------------------------------------------------
\begin{frame}[fragile]
  \frametitle{Future work}

Ideally, we would want to automatically parse the SQL queries and
determine which arguments should be quoted 

  \begin{itemize}
  \item A lot more work
  \item Would have to be done at load time for performance reasons
  \end{itemize}
  

\end{frame}



%-------------------------------------------------------------------------------
\begin{frame}[fragile]
  \frametitle{Questions}

  \begin{center}


{\Large
\texttt{dbapiext} is part of a \\
package named \texttt{antiorm}
}

\vfill

{\LARGE
antiorm homepage: \\
\verb=http://furius.ca/antiorm/=
}

\vfill

{\LARGE Questions?}

  \end{center}

\end{frame}


%===============================================================================
\end{document}

