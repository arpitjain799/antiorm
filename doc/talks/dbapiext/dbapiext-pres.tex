%
% Slides for given talks:
%
% * 2006-02-26: PyCon 2006, Dallas (Texas), USA
% * 2006-04-20: FACIL, Montreal (Quebec), Canada
%
%\documentclass[compress,trans]{beamer}
\documentclass{beamer}

\usepackage{graphicx}

\mode<presentation>
{
\usetheme{Singapore} % Singapore, default
  \setbeamertemplate{frametitle}[default][left]
  \setbeamercovered{transparent}
}

\usepackage[english]{babel}
\usepackage[latin1]{inputenc}

\usepackage{times}
\usepackage[T1]{fontenc}

\title{An extension to DBAPI for easier SQL queries}
\subtitle{}

\author{Martin Blais}

\institute{Furius Enterprise}

\date{PyCon 2007}

\subject{dbapiext presentation @ PyCon 2007}

\setcounter{tocdepth}{4}

\newcommand{\todo}[1]{}


\AtBeginSubsection[]
{
  \begin{frame}<beamer>
    \frametitle{Outline}
    \tableofcontents[currentsection,currentsubsection]
  \end{frame}
}


\begin{document}

%-------------------------------------------------------------------------------
\begin{frame}
  \titlepage
\end{frame}

%===============================================================================

% \begin{frame}
%   \frametitle{Intro - Desktop Search}
%
% So you have Google Desktop indexing your personal data on your computer\dots
% wouldn't it be great if this indexing database could be used to feed
% some of your blog automatically?
%
% Wouldn't it be awesome if your personal address book would be formed
% automatically by having a system find all the addresses in all of your
% documents?
%
% In this talk I will show a simple system that leverages docutils to do
% something like that.


%-------------------------------------------------------------------------------
\begin{frame}[fragile]
  \frametitle{Introduction}









\end{frame}


% %-------------------------------------------------------------------------------
% \begin{frame}[fragile]
%   \frametitle{1993 - Bookmarks}
% 
%   \begin{itemize}
%     \item Using Xmosaic
%     \item Creating lots of bookmarks lots of sites \\
%        (we did not have Google)
%   \end{itemize}
% 
% \emph{A script} was born, with typical input like this in a single \textbf{text
%   file}:
% 
% \begin{verbatim}
%   Raymond Hettinger's photography
%   http://www.knowyourboston.com
%   photography, boston, sexy girls
% \end{verbatim}
% 
%   \begin{itemize}
%     \item Then came Netscape, then came Mozilla, then came IE
%       \dots with corresponding converters.
% 
%     \item Eventually came Firefox and we were happy ever after\dots
%   \end{itemize}
% 
%   \hfill \dots or maybe not?
% 
% \end{frame}
% 
% 
% %% %-------------------------------------------------------------------------------
% %% \begin{frame}[fragile]
% %%   \frametitle{1993 - Bookmarks - Problems}
% %%
% %%   Problems I had with this system:
% %%   \begin{itemize}
% %%     \item A tree is inadequate for storing links (Bookmarks belong in many
% %%       ``groups''), and organizing by hand is an annoyance
% %%
% %%     \item You might want to share \textit{some} of these bookmarks \\
% %%       (e.g. del.icio.us)
% %%
% %%     \item You might want to search your bookmarks
% %%   \end{itemize}
% %%
% %% \vfill
% %%   \emph{Another script} was born\dots
% %%   \begin{itemize}
% %%   \item \texttt{tengis}: a small GUI app/database of bookmarks
% %%   \end{itemize}
% %%
% %% \end{frame}
% 
% 
% %-------------------------------------------------------------------------------
% \begin{frame}[fragile]
%   \frametitle{1997 - Address Book}
% 
%   \begin{itemize}
%     \item I was using ``paper'' technology to store my contact info - little
%       booklets\dots
% 
% \vfill
%     \item Then I used Netscape to store my contacts in LDIF
% 
% \vfill
%     \item One day, an old \texttt{nroff} user showed me this fabulous ``ascii''
%       technology to store his contact info:
% 
% {\small
% \begin{verbatim}
%       n: Librairie Michel Fortin inc.
%       a: 3714 St-Denis
%       p: +1.514.849.5719
% \end{verbatim}}
%   (He was a \verb=vi= user)
% 
%   \end{itemize}
% 
% \vfill
% 
%   Using this and ``paragraph-grep'' from a shell, I lived happily ever
%  after\dots
% 
% \vfill
% 
%   \hfill \dots or have I?
% 
% \end{frame}
% 
% 
% %-------------------------------------------------------------------------------
% \begin{frame}[fragile]
%   \frametitle{2000 - Blog}
% 
%   Those were the days before blogs were called blogs\dots
% 
%   \begin{itemize}
%   \item Pat Jennings/Synaptic - cycling through China
%   \item Phil Greenspun - \verb@photo.net@
%   \item I'm inspired!  \quad So I write \dots
%   \end{itemize}
% 
% \vfill\pause
% 
%   \dots \emph{another script}
%   \begin{itemize}
%     \item It takes its input in fashionable XML (it was truly awful)
%     \item So later I converted it to take input from \dots \emph{simple text
%       files}
% 
%     \item Eventually I discovered ReStructuredText and converted my system to
%           use it
% 
% %% Regenerating static pages is not fun, dynamic database-backed web
% %% sites are more interesting
%   \end{itemize}
% 
% \vfill
% 
%   And I was happy with static HTML files forever\dots \hfill \dots \emph{not!}
% 
% \end{frame}
% 
% 
% %-------------------------------------------------------------------------------
% \begin{frame}[fragile]
%   \frametitle{2002 - The Art of Taking Notes}
% 
%   I'm getting a little bit old now, I'm losing bits of memory\dots
% 
% \vfill
% 
%   But the older become smarter and now I just \textbf{know} in advance that I
%  will forget.  When I start a new task, I \textbf{invariably} start a new text
%  file to take notes on it.
% 
% \vfill
% 
%   This is great, because:
%   \begin{itemize}
%     \item I can grep the files
%     \item I can more easily interrupt my work \\
%       (there is a memory of the task)
%     \item I can put URLs in context, in these files, rather than in a global
%       bookmarks file
%   \end{itemize}
% 
% \end{frame}
% 
% 
% %-------------------------------------------------------------------------------
% \begin{frame}[fragile]
%   \frametitle{2002 - The Art of Taking Notes}
%   \framesubtitle{Wikis Suck (For This Purpose)}
% 
%   I would like to share many of these short technical documents with other
%   people \dots naturally, the idea of using Wikis come to mind.
% 
% \vfill
%   But wikis \emph{suck} for jotting down notes\dots
% 
% \begin{itemize}
% \item Anything but the most trivial topic title looks horrible:
% 
% \begin{verbatim}
%   BrazilTravelNotes
% \end{verbatim}
% 
% \item The editor capabilities of browsers are inadequate
%   \begin{itemize}
%   \item Who has never lost a file being edited in a TEXTAREA?
%   \item I'm a programmer, I want powerful editing!
%     I live in Emacs
%   \item I want to be able to save my files without having to submit
%   \end{itemize}
% \end{itemize}
% 
% \vfill\pause
% % * Cool idea: link an emacs instance within Firefox.
% % * Does not identify the meanings within the files either.
% 
% \end{frame}
% 
% 
% %-------------------------------------------------------------------------------
% \begin{frame}[fragile]
%   \frametitle{Mixed Data Example: Travel Files}
% 
%   Lots of scattered notes files\dots
% 
% \vfill
% 
%   One example of these notes files are my travel files, they contains many
%   different types of things:
% \begin{itemize}
% 
% \item They contain a list of things to do for a trip, personal notes,
%   itineraries
%   (\textbf{documents})
% 
% \item They contain addresses of people and places to visit
%   (\textbf{contact infos})
% 
% \item They contain URLs of related websites
%   (\textbf{bookmarks})
% 
% \item They contain references to books and articles
%   (\textbf{publications})
% 
% \end{itemize}
% 
% \end{frame}
% 
% 
% %-------------------------------------------------------------------------------
% \begin{frame}[fragile]
%   \frametitle{Mixed Data Example: Travel Files}
% 
% {\footnotesize
% \begin{verbatim}
% ====================
%    Trip to Brazil
% ====================
% 
% :Id: brazil-trip-notes
% :Category: Travel
% :Disclosure: public
% 
% Itinerary Proposals
% ===================
% 
%   Jan 25
%     * Fly to Salvador da Bahia, Brazil
% 
%   Jan 26
%     * Drink Caipirinhas
% \end{verbatim}
% }
% 
% \end{frame}
% 
% 
% %-------------------------------------------------------------------------------
% \begin{frame}[fragile]
%   \frametitle{Mixed Data Example: Travel Files}
% 
% {\footnotesize
% \begin{verbatim}
% Visa
% ====
% 
% * :n: Consulat g�n�ral du Br�sil � Montr�al
%   :a: 2000, rue Mansfield, bureau 1700, Montr�al (QC) H3A 3A5
%   :f: (514) 499-3963
%   :e: vistos@consbrasmontreal.org
%   :w: http://www.consbrasmontreal.org/
% 
% Vaccinations
% ============
% http://www.mdtravelhealth.com/destinations/samerica/brazil.html
% 
%   Routine immunizations
%     All travelers should be up-to-date on tetanus-diphtheria,
%     measles-mumps-rubella, polio, and varicella immunizations
% 
% \end{verbatim}
% }
% 
% \end{frame}
% 
% 
% %% %-------------------------------------------------------------------------------
% %% \begin{frame}[fragile]
% %%   \frametitle{Mixed Data Example: Travel Files}
% %%
% %% {\footnotesize
% %% \begin{verbatim}
% %%
% %% Accomodation
% %% ============
% %%
% %% * :n: �MBAR POUSADA
% %%   :a: Rua Afonso Celso, 485, Barra. Salvador - Bahia - Brasil
% %%   :p: 55-71-3264-6956 / 3267-1507
% %%   :e: ambarpousada@ambarpousada.com.br
% %%   :w: http://www.ambarpousada.com.br/
% %%
% %%   Para chegar
% %%
% %%   Do aeroporto, tem �nibus executivo "Pra�a da S�" via Farol da Barra durante o
% %%   dia, tem que saltar no Barra Center, na praia. Se voc� chegar � noite ou se
% %%   preferir o taxi, pe�am por e-mail e mandaremos, por sua conta, um dos nossos
% %%   taxistas preferidos!
% %% \end{verbatim}
% %% }
% %%
% %% \end{frame}
% 
% 
% %-------------------------------------------------------------------------------
% \begin{frame}[fragile]
%   \frametitle{Data Across Documents}
% 
%   My data is scattered \emph{across} the set of all my documents
% 
%   \includegraphics[width=1.0\textwidth]{across3.pdf}
% 
% \end{frame}
% 
% 
% %-------------------------------------------------------------------------------
% \begin{frame}[fragile]
%   \frametitle{Data Across Documents}
% 
%   What if I could \emph{identify} and \emph{extract} the meaningful parts from
%   those files and store them appropriately?  What could I build with this?
% 
% \vfill\pause
% 
%   Related idea: the Semantic Web
%   \begin{itemize}
%   \item In an ideal world, web page authors would identify all relevant parts of
%     their documents with appropriate markup
% 
%   \item You would then be able to collect and use this data, e.g. create an
%         ``address book'' of the internet
% 
%   \item Search engines are taking a stab at this holy grail
% 
%   \end{itemize}
% 
% \vfill
% 
%   But I want this \textbf{now}, and for just my personal corpus of files, even
%   if it's a restricted version of this idea.  I want a database built from the
%   files on my computer to feed a website, like a blog on steroids.
% 
% \end{frame}
% 
% 
% %-------------------------------------------------------------------------------
% \begin{frame}[fragile]
%   \frametitle{The Goal}
% 
%   Build a system that can extract semantically meaningful informations in my set
%   of personal info files, using weak heuristics and conventions, and store this
%   information in a structured way (in database tables), so I can use this
%   information later and serve it in new, interesting ways.
% 
% \vfill
% 
%   Nabu is a Python library that allows you to do that.
%   \begin{itemize}
%   \item It is not tied specifically to PIM info
%   \item You can write extractors for anything, you just have to establish
%     conventions for the docutils structures that you are going to
%     recognize/extract
%   \item On the client it requires only Python to publish text files \\
%     (minimize dependencies to allow easy deployment)
%   \end{itemize}
% 
% %   If I had served you this definition in the first place, I'm not sure you
% %   would be sitting here.
% 
% \end{frame}
% 
% 
% %-------------------------------------------------------------------------------
% \begin{frame}[fragile]
%   \frametitle{Components / Overview}
% 
% \includegraphics[width=1.0\textwidth]{../nabu2.pdf}
% 
% \end{frame}
% 
% 
% %-------------------------------------------------------------------------------
% \begin{frame}[fragile]
%   \frametitle{Components}
% 
%   \begin{itemize}
%   \item \textbf{Nabu Publisher Client}: searches the files on the client side
%     and sends the modified ones to the server
%     \begin{itemize}
%     \item It fetches MD5 sums from the server and compares the local files
%     \item Files are identified by an embedded Id, so file locations don't matter
% \begin{verbatim}
%    :Id: 8844db51-36ee-4e2a-8255-84e804f5cbe2
% \end{verbatim}
%     \end{itemize}
% 
% \pause
%   \item \textbf{Nabu Server}: receives the files, parses them through docutils
%     and runs the configured extractors, thereby storing the data
%     \begin{itemize}
%     \item All extracted data is tagged with the unique id for the file
%     \item When a file is uploaded, old data from that file is removed and new
%       data replaces it
%     \end{itemize}
% 
% \pause
%   \item \textbf{Storage}: typically, your database server \\
%     (or files or anything else if you like)
% 
% \pause
%   \item \textbf{Presentation}: your own favourite thing \\
%      (Nabu does not provide presentation)
% 
%   \end{itemize}
% 
% \end{frame}
% 
% 
% %-------------------------------------------------------------------------------
% \begin{frame}[fragile]
%   \frametitle{Design}
% 
% \includegraphics[width=0.9\textwidth]{design.pdf}
% 
% \end{frame}
% 
% 
% %-------------------------------------------------------------------------------
% \begin{frame}[fragile]
%   \frametitle{Extractors: Integration with docutils}
% 
%   Here is the complete docutils pipeline:
% 
% \vfill
% 
%   \includegraphics[width=1.0\textwidth]{docutils1.pdf}
% 
% \end{frame}
% 
% 
% %-------------------------------------------------------------------------------
% \begin{frame}[fragile]
%   \frametitle{Extractors: Integration with docutils}
% 
%   Here is the modified, partial docutils pipeline (no output, just process in
%   order to run the extractors):
% 
% \vfill
% 
%   \includegraphics[width=1.0\textwidth]{docutils2.pdf}
% 
% \end{frame}
% 
% 
% %-------------------------------------------------------------------------------
% \begin{frame}[fragile]
%   \frametitle{Extractors: docutils document tree}
% 
%   \texttt{docutils} is a ``2D parser''
%   \begin{itemize}
%   \item Its structures are recursive boxes of stuff
%   \item For your extractors you have to establish conventions for recognizing
%     the stuff you want to extract from your text files
%   \end{itemize}
% 
% \vfill
% 
% \begin{center}
%   \includegraphics[width=0.8\textwidth]{rest2d.png}
% 
% \vfill e.g. has name and (has email or has address)
% 
% \end{center}
% 
% 
% \end{frame}
% 
% 
% 
% %-------------------------------------------------------------------------------
% \begin{frame}[fragile]
%   \frametitle{Extractors: Viewing the docutils parse tree}
% 
%   You need to write extractors for the stuff that you are interested in, for
%   example:
% 
% {\scriptsize
% \begin{verbatim}
%      * :name: Bill Gates
%        :email: billg@microsoft.com
% \end{verbatim}
% }
% 
% \vfill
% 
%   Use \texttt{rst2pseudoxml.py} to figure how docutils parses it:
% 
% {\scriptsize
% \begin{verbatim}
% <bullet_list bullet="*">
%     <list_item>
%         <field_list>
%             <field>
%                 <field_name>
%                     name
%                 <field_body>
%                     <paragraph>
%                         Bill Gates
%             <field>
%                 <field_name>
%                     email
%                 <field_body>
%                     <paragraph>
%                         <reference refuri="mailto:billg@microsoft.com">
%                             billg@microsoft.com
% \end{verbatim}
% }
% 
% \end{frame}
% 
% 
% %-------------------------------------------------------------------------------
% \begin{frame}[fragile]
%   \frametitle{Extractors: Implementation}
% 
% {\small
% Then implement it:
% \begin{verbatim}
% class AddressExtractor(extract.Extractor):
% 
%     def apply( self, **kwargs ):
%         v = AddressVisitor(self, self.document)
%         self.document.walkabout(v)
% 
% class AddressVisitor(...):
%     ...
% 
% class AddressStorage(extract.SQLExtractorStorage):
% 
%     def store( self, unid, name, tfields ):
%         ... # store the stuff in a database
% 
% \end{verbatim}
% %    default_priority = 900
% }
% 
% \end{frame}
% 
% 
% %-------------------------------------------------------------------------------
% \begin{frame}[fragile]
%   \frametitle{Target Audience}
% 
% {\large\begin{center}
% \textbf{It's not for your mom!}
% \end{center}}
% 
%   Nabu is intended only for use by people who have developed the \emph{ability
%     to edit text files carefully} (typically programmers, i.e. you guys).
% 
%  \begin{itemize}
%  \item We understand indentation
%  \item We know about spacing, justification, filling, etc.
%  \item We are careful about pesky little details
%  \item This is what makes creating ReST files possible
%  \end{itemize}
% 
% \vfill
% 
% \begin{center}
%   Leverage this ability!
% \end{center}
% 
% \end{frame}
% 
% 
% %-------------------------------------------------------------------------------
% \begin{frame}[fragile]
%   \frametitle{Presentation: Document Example}
% 
%   \begin{center}
%     \includegraphics[width=0.9\textwidth]{page-shot.pdf}
%   \end{center}
% 
% \end{frame}
% 
% 
% %-------------------------------------------------------------------------------
% \begin{frame}[fragile]
%   \frametitle{Presentation: Peepholes}
% 
% \vfill
% 
%   \begin{center}
%     \includegraphics[width=0.9\textwidth]{peephole1.pdf}
%   \end{center}
% 
% \vfill
% \hrule
% \vfill
% 
%   \begin{center}
%     \includegraphics[width=0.9\textwidth]{peephole2.pdf}
%   \end{center}
% 
% \vfill
% 
% \end{frame}
% 
% %-------------------------------------------------------------------------------
% \begin{frame}[fragile]
%   \frametitle{Presentation: Peepholes}
% 
%   Peepholes are passwordless privileged acccess to selected documents in my Nabu
%   store.
%   \begin{itemize}
%   \item Useful for sending a link to a file that is work-in-progress
%   \item Does not require users/passwords, relies on the ``relative'' privacy of
%     email
%   \item Provides access to only some selected resources
%   \item Expires after some time, or after a number of accesses (self-destroying
%     links\dots [insert M.I. music here])
%   \end{itemize}
% \end{frame}
% 
% 
% %-------------------------------------------------------------------------------
% \begin{frame}[fragile]
%   \frametitle{Presentation: Events/Calendar Example}
% 
% {\small
% \begin{verbatim}
% sat 2005-12-31 19h30
%   - NYE Evening chez Stuart
% 
% sat 2005-12-31
%   - Proteus: mkisofs for backup copy DVD-ROM
% 
% 2006-01-02
%   - Contact SonoMax about free earplugs
%   - Track Leif's grille that was supposed to arrive.
% 
% 2006-01-03
%   - Visit Yves * D'ailleurs je vais avoir besoin de tes info pour les
%     salaires de 2005
% 
%   - Confirm flight to Brazil w/ Constellation
% 
% 2006-01-04 20h00
%   - Dinner w/ Pierre @ Golden Cari
% 
% 2006-01-05
%   - Book room for PyCon (should be 79 USD) in january when problems are fixed.
% 
% 2006-01-13, 14, 16
%   - Vote par anticipation
% 
% \end{verbatim}
% }
% 
% \end{frame}
% 
% 
% %-------------------------------------------------------------------------------
% \begin{frame}[fragile]
%   \frametitle{Presentation: Events/Calendar Example}
%   \framesubtitle{Calendar View}
% 
% \includegraphics[width=1.0\textwidth]{calendar-shot.pdf}
% 
% \end{frame}
% 
% 
% %-------------------------------------------------------------------------------
% \begin{frame}[fragile]
%   \frametitle{Presentation: Other Examples}
% 
%   \begin{itemize}
%   \item Bookmarks: You can serve your extracted bookmarks from a server using
%     RSS format
% 
%   \item Billing info: you could define a simple timesheet format in a text file
%     and have some kind per-task or per-client billing info page that is always
%     up-to-date
% 
%   \item Filter all extracted links to Google Maps and generate a map of all of
%     the locations with links to the original documents
% 
%   \item \dots (insert your own here) \dots
% 
%   \item Dynamic website data: you could use Nabu to feed some data for a web
%     site, this allows you to \textbf{avoid having to write input forms}
% 
%   \end{itemize}
% 
% \end{frame}
% 
% 
% %-------------------------------------------------------------------------------
% \begin{frame}[fragile]
%   \frametitle{Debugging: Nabu Contents Browser}
%   \framesubtitle{View Uploaded File Details}
% 
% \includegraphics[width=1.0\textwidth]{ll-upload.pdf}
% 
% \end{frame}
% 
% 
% %-------------------------------------------------------------------------------
% \begin{frame}[fragile]
%   \frametitle{Debugging: Nabu Contents Browser}
%   \framesubtitle{View Extracted Info}
% 
% \includegraphics[width=1.0\textwidth]{ll-extracted.pdf}
% 
% \end{frame}
% 
% 
% %-------------------------------------------------------------------------------
% \begin{frame}[fragile]
%   \frametitle{Multiple Users}
% 
%   There are two approches:
% 
%   \begin{enumerate}
%   \item Multiple users share a single body of files
%     \begin{itemize}
%     \item You could update to Nabu on a \textbf{Subversion hook} when files are
%       committed
%     \end{itemize}
% 
%   \item Each user has a distinct body of files
%     \begin{itemize}
%     \item The Nabu server supports disjoint sets of ids per-user, so that users
%       don't have to manually manage avoiding collisions
%     \end{itemize}
% 
%   \end{enumerate}
% 
% \end{frame}
% 
% 
% 
% %-------------------------------------------------------------------------------
% \begin{frame}[fragile]
%   \frametitle{Problems}
% 
%   \begin{itemize}
%   \item Creating precise reStructuredText can be fragile from the user's
%     point-of-view
% 
%     \begin{itemize}
%     \item You end up having to wrap your head around the docutils document
%       structure and knowing ReST really well in order to generate what you need
%       (but that is ok)
%     \end{itemize}
% 
% \vfill
% 
%   \item You need to make sure that your set of ids are unique
%     \begin{itemize}
%     \item I like to use UUIDs like this:
%        \verb=  78d8a600-abb4-49c0-a0fc-1c315cddbc1a  =
%     \end{itemize}
% 
%   \end{itemize}
% 
% \end{frame}
% 
% 
% %-------------------------------------------------------------------------------
% \begin{frame}[fragile]
%   \frametitle{Future Work}
% 
%   \begin{itemize}
% 
%   \item Stabilize more
%     \begin{itemize}
%     \item I need to write more presentations for my data
%     \item Convert presentations output to microformats
%     \end{itemize}
% 
%   \item Support encryption in the publisher client, hidden files
% 
%   \item Support per-document options
% 
%   \end{itemize}
% 
% \end{frame}


%-------------------------------------------------------------------------------
\begin{frame}[fragile]
  \frametitle{Questions}

  \begin{center}

{\LARGE
Nabu homepage: \\
\verb=http://furius.ca/nabu/=
}

\vfill

Slides will be posted there. \\
I will be around during the sprints.

\vfill

{\LARGE Questions?}

  \end{center}

\end{frame}


%===============================================================================
\end{document}

